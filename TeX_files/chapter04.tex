\chapter{A Classical Scalar Field}

From classical physics, interaction between bodies seems to be instantaneous. Consider the Coulomb's Law describing the interaction between charged particles $q_1$ and $q_2$ separated by a distance $r$ in the direction $\hat{r}$. If we change the position of particle 1 a little bit, then particle 2 instantaneously feels the effects according to $\textbf{F}_e=(kq_1q_2/r^2)\hat{r}$. This is problematic, for the reason that causal influences cannot travel faster than light -- following principles of special relativity. Thus, for relativistically valid laws of physics, interaction cannot be instantaneous. One successful solution to this problem is to describe the interaction of the charged particles in terms of the mediating electromagnetic field. Instead of particle 1 instantaneously affecting particle 2, we can think that the change in particle 1 creates a local disturbance in the electromagnetic field surrounding the particle, and this distrubance travels away from particle 1's position at the speed of light, eventually reaching particle 2. We can use the Maxwell's equation to describe this propagation of electromagnetic fields, which is surprisingly consistent with principles of relativity. This approach also applies to analogous problems in quantum mechanics. We will find that a relativistic theory of quantum mechanics not only requires interactions to be described by fields, but particles themselves must be describe in terms of fields.

We might think that we have found the ultimate answer to make any physical laws relativistically valid. However, the field description of even the simplest realistic particle, such as the electron, is not straightforward to be obtained. It is then instructive to start with the simplest case in terms of mathematical complexity involved. We consider a scalar field, which assigns a frame-independent number at every point in space and time. Although, take note that no low-mass elementary particle is described by such a field, aside from the Higgs Boson.

\section{The Principle of Least Action}
